\documentclass[
  english,            % define the document language (english, german)
  aspectratio=169,    % define the aspect ratio (169, 43)
  % handout=2on1,       % create handout with multiple slides (2on1, 4on1)
  % partpage=false,     % insert page at beginning of parts (true, false)
  % sectionpage=true,   % insert page at beginning of sections (true, false)
]{tumbeamer}


% load additional packages
\usepackage{booktabs}
\usepackage{amssymb}
\usepackage{mdframed}
\usepackage{amsthm}
\usepackage{mathtools}
\usepackage{multicol}
\usepackage{hyperref}
%\usepackage{julialogo}

\makeatletter
\patchcmd{\@Aboxed}{\boxed{#1#2}}{\fcolorbox{white}{blue!20}{$\displaystyle #1#2$}}{}{}%
\makeatother

\newtheorem{theorem}{Theorem}
\newtheorem{lemma}{Lemma}
\newtheorem{definition}{Definition}
\newtheorem{proposition}{Proposition}

%\usepackage{courier}
\usepackage{xcolor}
\usepackage{tikz}
\usepackage{tikzit}
\usetikzlibrary{shapes,arrows,calc,decorations.pathreplacing,angles,quotes}

\input{colors.tikzstyles}

\usepackage{empheq}
\usepackage{tcolorbox}

\tcbset{highlight math style={colback=blue!20!white,arc=2pt,boxrule=0pt}}

\usepackage{algpseudocodex}
\usepackage{algorithm}


%\colorlet{lightblue}{!10}
\usepackage{listings}
\lstset{
	basicstyle=\ttfamily,
	backgroundcolor = \color{blue!10},
	keywordstyle=\color{blue},
    numberstyle=\tiny\color{codegray},
    stringstyle=\color{purple},
    basicstyle=\ttfamily\footnotesize,
	% numbers=left, 
	numberstyle={\footnotesize \color{blue!50}},
	xleftmargin=.1in,
	numbersep=3pt,
	literate={\$}{{\textcolor{blue}{\$}}}1,
  showlines=true 
}
	
\usepackage{lmodern}

\usepackage{fontawesome}
\usepackage{julialogo}

\usepackage{nameref}
\makeatletter
\newcommand*{\currentname}{\@currentlabelname}
\makeatother
%\usepackage{dsfont}

\newcommand{\R}{{\mathbb R}}
\newcommand{\N}{{\mathbb N}}
\newcommand{\E}{{\mathbb E}}
\renewcommand{\P}{{\mathbb P}}
\newcommand{\bbC}{{\mathbb C}}
\newcommand{\cO}{\mathcal{O}}
\newcommand{\cF}{\mathcal{F}}
\newcommand{\cE}{\mathcal{E}}
\newcommand{\cX}{\mathcal{X}}
\newcommand{\cA}{\mathcal{A}}
\newcommand{\cS}{\mathcal{S}}
\newcommand{\cR}{\mathcal{R}}
\newcommand{\cB}{\mathcal{B}}
\newcommand{\cP}{\mathcal{P}}

\newcommand{\tV}{\widetilde{V}}
\newcommand{\tF}{\widetilde{F}}
\newcommand{\tQ}{\widetilde{Q}}

\newcommand{\bw}{\mathbold{w}}

\newcommand{\step}[1]{\mathds{1}_{\{#1\ge 0\}}}

\newcommand{\eps}{\varepsilon}

\renewcommand{\emph}[1]{\textcolor{purple}{#1}}
\newcommand{\argmax}{\mathop{\textrm{argmax}}}
\newcommand{\mean}{\mathop{\textrm{mean}}}
\newcommand{\diam}{\mathop{\textrm{diam}}}


\setbeamertemplate{itemize items}[circle]

% presentation metadata
\title{GAIO.jl}
\subtitle{Fast, Elegant Set-Oriented Numerical Methods for Dynamical Systems}
\author{April Herwig}

\institute{\theChairName\\\theDepartmentName\\\theUniversityName}
\date{}

%\footline{\insertauthor~|~\insertshorttitle~|~\insertshortdate}
\footline{}


% macro to configure the style of the presentation
\TUMbeamersetup{
  title page = TUM centered,         % style of the title page
  part page = TUM toc,            % style of part pages
  section page = TUM toc,         % style of section pages
  content page = TUM more space,  % style of normal content pages
  tower scale = 1.0,              % scaling factor of TUM tower (if used)
  headline = TUM empty,      % which variation of headline to use
  footline = TUM default,         % which variation of footline to use
  % configure on which pages headlines and footlines should be printed
  headline on = {title page},
  footline on = {every page, title page=false},
}

% available frame styles for title page, part page, and section page:
% TUM default, TUM tower, TUM centered,
% TUM blue default, TUM blue tower, TUM blue centered,
% TUM shaded default, TUM shaded tower, TUM shaded centered,
% TUM flags
%
% additional frame styles for part page and section page:
% TUM toc
%
% available frame styles for content pages:
% TUM default, TUM more space
%
% available headline options:
% TUM empty, TUM oneliner, TUM twoliner, TUM threeliner, TUM logothreeliner
%
% available footline options:
% TUM empty, TUM default, TUM infoline

\setlength{\parskip}{1em}

\usepackage{mathpple}
%\usepackage{euler}
%\usepackage{palatino}

\begin{document}

\tikzstyle{block} = [thick, draw, rectangle, rounded corners, fill=blue!20,
                       minimum height=3em, minimum width=6em]
\tikzstyle{node} = [thick, draw, circle, fill=blue!20]
\tikzstyle{action} = [thick, draw, circle, fill=red!10]

\maketitle

\begin{frame}{Dynamical systems}

Consider a continuous map
\[
f:\R^d\to \R^d.
\] 

The map defines a \emph{discrete dynamical system} by iteration:
\[
x_{k+1} = f(x_k), \qquad k=0,1,2,\ldots
\]

\begin{tcolorbox}[colback=blue!5!white,colframe=blue!75!black]
Basic question: \textit{What is the fate of some $x_0$ as $k\to\infty$?}
\end{tcolorbox}



\end{frame}

\begin{frame}{Attractors}

\begin{definition}
    A set $A\subset\R^d$ is \emph{invariant}, if
    \[
    f^{-1} (A) = A.
    \]
\end{definition}

\begin{tcolorbox}[colback=blue!5!white,colframe=blue!75!black]
Example: $A=\{\bar x\}$ with $\bar x = f(\bar x)$ a \emph{fixed point}.
\end{tcolorbox}


\begin{definition}
    An invariant set $A$ is \emph{attracting} if there is a neighborhood $U$ of $A$ such that for every open set $V\supset A$ there is $K\in\N$ such that
    \[
    f^k(U)\subset V \quad\text{for all } k\ge K.
    \]
\end{definition}

\end{frame}

\begin{frame}{Relative attractors}

\begin{proposition}
    If $A$ is a closed attracting set then
    \[
    A = \bigcap_{k\in\N} f^k(U).
    \]
\end{proposition}

\begin{definition}
    For some compact set $Q\subset\R^d$, the \emph{attractor relative to} $Q$ is
    \[
    A_Q \overset{def}{=} \bigcap_{k\in\N} f^k(Q).
    \]    
\end{definition}

We have
\begin{itemize}
    \item $A_Q$ is not necessarily invariant,
    \item but any invariant subset of $Q$ is contained in $A_Q$.  
\end{itemize}

\end{frame}

\begin{frame}{Computing $A_Q$}{The subdivision algorithm}

Generate a sequence $\cB_0,\cB_1,\cB_2,\ldots$ of finite families of compact sets as follows:

Let $\cB_0=\{Q\}$, $\theta \in (0,1)$. For $k=1,2,\ldots$ do
\begin{itemize}
    \item construct $\hat\cB_k$ such that
    \[
    |\hat\cB_k| = |\cB_{k-1}|
    \quad\text{and}\quad \diam\hat\cB_k \leq \theta \cdot \diam\cB_{k-1}.
    \]
    \item set
    \begin{gather*}
    \cB_k = f(\cB_k), \quad \text{where}\quad \\ 
    f(\cB_k) \overset{def}{=} \{ B\in\hat\cB_k \mid \exists B'\in\hat\cB_k: f^{-1}(B)\cap B' \neq \varnothing\}.
    \end{gather*}
\end{itemize}

\medskip

\begin{theorem}
$|\cB_k|\to A_Q$ as $k\to\infty$ in the Hausdorff metric.
\end{theorem}

\end{frame}

\begin{frame}{Computing $A_Q$}{Cell Mapping}

So how do we compute 
\[
  f(\cB_k) = \{ B\in\hat\cB_k \mid \exists B'\in\hat\cB_k: f^{-1}(B)\cap B' \neq \varnothing\}\ ?
\]

\begin{multicols}{2}

\begin{figure}
  \ctikzfig{boximage}
  %\caption{Image of the simple box set $\cB = \left\{ B \right\}$}
  \label{fig:boximage}
\end{figure}

\columnbreak

\vspace*{2ex}
\begin{itemize}
  \item test-point sampling \vspace*{1ex}
  \item interval arithmetic
\end{itemize}

\end{multicols}

\end{frame}

\begin{frame}[fragile]
\frametitle{The {\LARGE \julia} programming language}

Vision: an \emph{open-source}, \emph{elegant}, and \emph{fast} language for scientific computing

\begin{itemize}
  \item Composable, uses \emph{Multiple-dispatch} paradigm \vspace*{1ex}
  \item Easy to read \vspace*{1ex}
  \item Many built-in standard libraries for scientific computing \vspace*{1ex}
  \item Well documented from the beginning \vspace*{1ex}
\end{itemize}

%\begin{lstlisting}[language=Matlab,mathescape]
%julia> # unicode input natively supported
%       $\bbC$ = Complex
%Complex
%
%julia> # mutiple dispatch: Base functions can be extended
%       Base.in(z, ::Type{Complex}) = z isa Complex
%
%julia> z = 5.0 + 3.0*im
%5.0 + 3.0*im
%
%julia> z $\in$ $\bbC$
%true
%\end{lstlisting}

\end{frame}

\begin{frame}[fragile]
\frametitle{GAIO in {\LARGE \julia}: GAIO.jl}

%\medskip

Compare the pseudocode algorithm
  
\begin{algorithmic}[1]
    %\Function{relative_attractor}{$f$, $\cB$, steps}
    \Require $f,\,\ \cB_0,\,\ n_{\text{steps}}$
    %\State $\mathcal{B}_0 \gets \mathcal{B}$

    \For{$k = \left\{ 1,\ \dotsc,\ n_{\text{steps}} \right\}$}
        %\Comment{$n$ is a predefined number of iteration steps}
        \State $\mathcal{B}_k \gets$ \Call{subdivide}{$\,\mathcal{B}_{k-1}\,$}
        \State $\mathcal{B}_k \gets \mathcal{B}_k\, \cap\, f (\,\mathcal{B}_k\,)$
    \EndFor

    \State \Return $\mathcal{B}_n$ 
    %\EndFunction
\end{algorithmic}

to the julia implementation

\begin{lstlisting}[language=Matlab,mathescape]
  function relative_attractor(f::BoxMap, $\cB$::BoxSet, steps)
    for k = 1:steps
      $\cB$ = subdivide($\cB$)
      $\cB$ = $\cB$ $\cap$ f($\cB$)
    end
    return $\cB$
  end
\end{lstlisting}

\end{frame}

\begin{frame}{GAIO in {\LARGE \julia}: GAIO.jl}{demos}

\begin{itemize}
  \item Relative attractor of the Hénon map \vspace*{2ex}
  \item Invariant probability measure in the Lorenz system \vspace*{2ex}
  \item Almost invariant (metastable) sets in Chua's circuit \vspace*{2ex}
\end{itemize}

\end{frame}

\begin{frame}{GAIO in {\LARGE \julia}: GAIO.jl}{other (current and future) examples}

\begin{itemize}
  \item Coherent sets using the technique of Froyland \& Padberg (2009)
  \item Topological entropy (by next week or I owe you a beer)
  \item Conley index pair generation
  \item Morse graph / decomposition and adherence
\end{itemize}

These (and many more) can be found in GAIO.jl's documentation at \url{gaioguys.github.io/GAIO.jl/}  
  
\end{frame}


\end{document}
