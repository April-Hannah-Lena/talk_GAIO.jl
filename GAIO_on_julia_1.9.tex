\documentclass[
  english,            % define the document language (english, german)
  aspectratio=169,    % define the aspect ratio (169, 43)
  % handout=2on1,       % create handout with multiple slides (2on1, 4on1)
  % partpage=false,     % insert page at beginning of parts (true, false)
  % sectionpage=true,   % insert page at beginning of sections (true, false)
]{tumbeamer}


% load additional packages
\usepackage{booktabs}
\usepackage{amssymb}
\usepackage{mdframed}
\usepackage{amsthm}
\usepackage{mathtools}
\usepackage{multicol}
\usepackage{hyperref}
%\usepackage{julialogo}

\makeatletter
\patchcmd{\@Aboxed}{\boxed{#1#2}}{\fcolorbox{white}{blue!20}{$\displaystyle #1#2$}}{}{}%
\makeatother

\newtheorem{theorem}{Theorem}
\newtheorem{lemma}{Lemma}
\newtheorem{definition}{Definition}
\newtheorem{proposition}{Proposition}

%\usepackage{courier}
\usepackage{xcolor}
\usepackage{tikz}
\usepackage{tikzit}
\usetikzlibrary{shapes,arrows,calc,decorations.pathreplacing,angles,quotes}

\input{colors.tikzstyles}

\usepackage{empheq}
\usepackage{tcolorbox}

\tcbset{highlight math style={colback=blue!20!white,arc=2pt,boxrule=0pt}}
\newenvironment{emphbox}
  {\begin{tcolorbox}[colback=blue!5!white,colframe=blue!75!black]}
  {\end{tcolorbox}}


\usepackage{algpseudocodex}
\usepackage{algorithm}

%\colorlet{lightblue}{!10}
\usepackage{listings}
\lstset{
	basicstyle=\ttfamily,
	backgroundcolor = \color{blue!10},
	keywordstyle=\color{blue},
    numberstyle=\tiny\color{codegray},
    stringstyle=\color{purple},
    basicstyle=\ttfamily\footnotesize,
	% numbers=left, 
	numberstyle={\footnotesize \color{blue!50}},
	xleftmargin=.1in,
	numbersep=3pt,
	literate={\$}{{\textcolor{blue}{\$}}}1,
  showlines=true 
}
	
\usepackage{lmodern}

\usepackage{fontawesome}
\usepackage{julialogo}

\usepackage{nameref}
\makeatletter
\newcommand*{\currentname}{\@currentlabelname}
\makeatother
%\usepackage{dsfont}

\newcommand{\R}{{\mathbb R}}
\newcommand{\N}{{\mathbb N}}
\newcommand{\E}{{\mathbb E}}
\renewcommand{\P}{{\mathbb P}}
\newcommand{\bbC}{{\mathbb C}}
\newcommand{\cO}{\mathcal{O}}
\newcommand{\cF}{\mathcal{F}}
\newcommand{\cE}{\mathcal{E}}
\newcommand{\cX}{\mathcal{X}}
\newcommand{\cA}{\mathcal{A}}
\newcommand{\cS}{\mathcal{S}}
\newcommand{\cR}{\mathcal{R}}
\newcommand{\cB}{\mathcal{B}}
\newcommand{\cP}{\mathcal{P}}

\newcommand{\tV}{\widetilde{V}}
\newcommand{\tF}{\widetilde{F}}
\newcommand{\tQ}{\widetilde{Q}}

\newcommand{\bw}{\mathbold{w}}

\newcommand{\step}[1]{\mathds{1}_{\{#1\ge 0\}}}

\newcommand{\eps}{\varepsilon}

\renewcommand{\emph}[1]{\textcolor{purple}{#1}}
\newcommand{\argmax}{\mathop{\textrm{argmax}}}
\newcommand{\mean}{\mathop{\textrm{mean}}}
\newcommand{\diam}{\mathop{\textrm{diam}}}


\setbeamertemplate{itemize items}[circle]

% presentation metadata
\title{GAIO.jl}
\subtitle{Preparing for a 1.0 release using julia 1.9}
\author{April Herwig}

\institute{\theChairName\\\theDepartmentName\\\theUniversityName}
\date{}

%\footline{\insertauthor~|~\insertshorttitle~|~\insertshortdate}
\footline{}


% macro to configure the style of the presentation
\TUMbeamersetup{
  title page = TUM centered,         % style of the title page
  part page = TUM toc,            % style of part pages
  section page = TUM toc,         % style of section pages
  content page = TUM more space,  % style of normal content pages
  tower scale = 1.0,              % scaling factor of TUM tower (if used)
  headline = TUM empty,      % which variation of headline to use
  footline = TUM default,         % which variation of footline to use
  % configure on which pages headlines and footlines should be printed
  headline on = {title page},
  footline on = {every page, title page=false},
}

% available frame styles for title page, part page, and section page:
% TUM default, TUM tower, TUM centered,
% TUM blue default, TUM blue tower, TUM blue centered,
% TUM shaded default, TUM shaded tower, TUM shaded centered,
% TUM flags
%
% additional frame styles for part page and section page:
% TUM toc
%
% available frame styles for content pages:
% TUM default, TUM more space
%
% available headline options:
% TUM empty, TUM oneliner, TUM twoliner, TUM threeliner, TUM logothreeliner
%
% available footline options:
% TUM empty, TUM default, TUM infoline

\setlength{\parskip}{1em}

\usepackage{mathpple}
%\usepackage{euler}
%\usepackage{palatino}

\begin{document}

\tikzstyle{block} = [thick, draw, rectangle, rounded corners, fill=blue!20,
                       minimum height=3em, minimum width=6em]
\tikzstyle{node} = [thick, draw, circle, fill=blue!20]
\tikzstyle{action} = [thick, draw, circle, fill=red!10]

\maketitle

\begin{frame}{Motivation}{Attractors}

\begin{definition}
    An invariant set $A$ is \emph{attracting} if there is a neighborhood $U$ of $A$ such that for every open set $V\supset A$ there is $K\in\N$ such that
    \[
    f^k(U)\subset V \quad\text{for all } k\ge K.
    \]
\end{definition}

\begin{proposition}
    If $A$ is a closed attracting set then
    \[
    A = \bigcap_{k\in\N} f^k(U).
    \]
\end{proposition}

\begin{emphbox}
    \textbf{Basic idea:} Successively refine an approximation of $A$ using \emph{subdivision}
\end{emphbox}

\end{frame}

\begin{frame}{Computing Attractors}{The subdivision algorithm}

Generate a sequence $\cB_0,\cB_1,\cB_2,\ldots$ of finite families of compact sets as follows:

Let $\cB_0=\{Q\}$, $\theta \in (0,1)$. For $k=1,2,\ldots$ do
\begin{itemize}
    \item construct $\hat\cB_k$ such that
    \[
    |\hat\cB_k| = |\cB_{k-1}|
    \quad\text{and}\quad \diam\hat\cB_k \leq \theta \cdot \diam\cB_{k-1}.
    \]
    \item set
    \begin{gather*}
    \cB_k = f(\hat\cB_k) \cap \hat\cB_k, \quad \text{where}\quad \\ 
    f(\hat\cB_k) \cap \hat\cB_k \,\overset{def}{=}\, \{ B\in\hat\cB_k \mid \exists B'\in\hat\cB_k: f(B')\cap B \neq \varnothing\}.
    \end{gather*}
\end{itemize}

\medskip

\begin{emphbox}
  \begin{theorem}
    $|\cB_k|\to A_Q$ as $k\to\infty$ in the Hausdorff metric.
  \end{theorem}
\end{emphbox}

\end{frame}

\begin{frame}{Computing Attractors}{The subdivision algorithm}
  
\begin{figure}
  \includegraphics[height=0.8\textheight]{figures/henon_subdivisions}
\end{figure}

\end{frame}

\begin{frame}[fragile]
\frametitle{Representation of Cubical Complexes}

\texttt{BoxPartition}: partition the domain into equally sized grid of hypercubes, or "Boxes"

\begin{lstlisting}[language=Matlab,mathescape]
  struct BoxPartition{N,T,I<:Integer}
      domain::Box{N,T}
      dims::SVector{N,I}
  end
\end{lstlisting}

\texttt{TreePartition}: binary tree holding successive subdivisions

\begin{lstlisting}[language=Matlab,mathescape]
  struct Node{I<:Integer}
      left::I
      right::I
  end

  struct TreePartition{N,T,I,V<:AbstractArray{Node{I}}}
      domain::Box{N,T}
      nodes::V
  end
\end{lstlisting}

\end{frame}

\begin{frame}[fragile]
\frametitle{Representation of Cubical Complexes}

\texttt{BoxSet}: collections of Boxes within a partition

\begin{lstlisting}[language=Matlab,mathescape]
  struct BoxSet{B,P<:AbstractBoxPartition{B},S<:AbstractSet} <: AbstractSet{B}
      partition::P
      indices::S
  end
\end{lstlisting}

We can use the built-in set data types and setwise operations for \texttt{BoxSet}s using \emph{multiple-dispatch}

\begin{lstlisting}[language=Matlab,mathescape]
  function Base.$\subseteq$(B$_1$::BoxSet, B$_2$::BoxSet)
      ( B$_1$.partition == B$_2$.partition ) && ( B$_1$.indices $\subseteq$ B$_2$.indices )
  end
\end{lstlisting}

\end{frame}

\begin{frame}{Cell Mapping}

So how do we compute 
\[
  f(\cB) = \{ B\in\cP \mid \exists B'\in\cB: f(B')\cap B \neq \varnothing\}\ ?
\]

\begin{figure}
  \ctikzfig{boximage}
  %\caption{Image of the simple box set $\cB = \left\{ B \right\}$}
  \label{fig:boximage}
\end{figure}

\end{frame}

\begin{frame}[fragile]
\frametitle{Cell Mapping}
\framesubtitle{test point sampling}

\medskip

\begin{lstlisting}[language=Matlab,mathescape]
function map_boxes(g::SampledBoxMap, source::BoxSet)

  B() = empty(source)                # Function to initialize empty BoxSet
  P = source.partition

  $\text{\textcolor{blue}{\texttt{@floop}}}$ for box in source
    for p in g.domain_points(box)    # Generate sample points
      fp = typesafe_map(g, p)        # Wrap user-function output
      hit = cover(P, fp)             # Box in P covering the point fp
      $\text{\textcolor{blue}{\texttt{@reduce}}}$(image = B() $\cup$ hit) $\quad\ \ \,$   # Each thread collects hits,
    end                              # after loop completion the 
  end                                # result is reduced

  return image
end 
\end{lstlisting}

\end{frame}

\begin{frame}{Cell Mapping}{test point sampling}
  
\begin{itemize}
  \item choose test points within $\cB$ and record their images under $f$
  \item memory-efficient "lazy" test point sampling with \texttt{Generator}s
  \item ensure type-stability to shorten generated code
  \item spread load across multiple compute threads using \textcolor{blue}{\texttt{@floop}} macro
  \item collect hits and reduce per-thread result into single result using \textcolor{blue}{\texttt{@reduce}} macro
  \item can harness the GPU using CUDA.jl
\end{itemize}

\end{frame}

\begin{frame}[fragile]
\frametitle{Cell Mapping}

All \texttt{BoxMap} types work under a common API: they must define 

\begin{lstlisting}[language=Matlab,mathescape]
  map_boxes(g::MyBoxMap, source::BoxSet)
  construct_transfers(g::MyBoxMap, domain::BoxSet)
  construct_transfers(g::MyBoxMap, domain::BoxSet, codomain::BoxSet)
\end{lstlisting}

You now have all the knowledge to understand \texttt{TransferOperator, BoxGraph} as well! 
  
\end{frame}

\begin{frame}[fragile]
\frametitle{Optimization: Solving "Time-To-First-Attractor"}
  
Investigating the first execution shows an interesting problem: a GPU kernel gets compiled... even when no GPU is present?

Solve by converting the CUDA dependency to an \emph{extension}

\begin{lstlisting}[language=Matlab,mathescape]
  name = "GAIO"
  uuid = "33d280d1-ac47-4b0f-9c2e-fa6a385d0226"
  authors = ["The GAIO.jl Team"]
  version = "1.0.0"

  [deps]
  ...

  [weakdeps]
  CUDA = "052768ef-5323-5732-b1bb-66c8b64840ba"
  ...

  [extensions]
  CUDAExt = "CUDA"
  ...
\end{lstlisting}

\end{frame}

\begin{frame}[fragile]
\frametitle{Optimization: Solving "Time-To-First-Attractor"}

With dynamic dispatch reduced, we can now save precompiled code in a \emph{package image}. To force precompilation of common workloads, use \texttt{PrecompileTools.jl}

\medskip

\begin{lstlisting}[language=Matlab,mathescape]
  using PrecomileTools

  $\text{\textcolor{blue}{\texttt{@setup\_workload}}}$ begin
    # set local variables for common workload, e.g. attractor of Henon map

    $\text{\textcolor{blue}{\texttt{@compile\_workload}}}$ begin
      # track which code gets generated and force full compilation
    end

  end
\end{lstlisting}

\end{frame}


\end{document}
