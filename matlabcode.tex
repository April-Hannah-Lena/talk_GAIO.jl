\documentclass[12pt,a4paper,twoside]{article}
\usepackage{import}
\usepackage{amsmath}
\usepackage{amssymb}
\usepackage{amsthm}
\usepackage{amsfonts}
\usepackage{float}
\usepackage{hyperref}
\usepackage{multicol}
\usepackage{graphicx}
\usepackage{geometry}
\usepackage[utf8]{inputenc}
\usepackage{caption}
\usepackage{subcaption}
\usepackage{afterpage}
\usepackage[section]{placeins}
\usepackage{biblatex}
\usepackage{fontspec}
\usepackage{setspace}
\usepackage{color}
\usepackage[noEnd=false]{algpseudocodex}
\usepackage{algorithm}
\usepackage{tikz}
\usepackage{tikzit}
\usepackage{julialogo}
\usepackage[autoload=false, linenumbers=true, theme=default-plain]{jlcode}

\definecolor{mygreen}{RGB}{28,172,0} % color values Red, Green, Blue
\definecolor{mylilas}{RGB}{170,55,241}
\definecolor{mygrey}{HTML}{888888}
\lstdefinestyle{mcodestyle}{%
    basicstyle=\ttfamily\relsize{-1},
    breaklines=true,%
    morekeywords={matlab2tikz},
    keywordstyle=\color{blue},%
    morekeywords=[2]{1}, keywordstyle=[2]{\color{black}},
    identifierstyle=\color{black},%
    stringstyle=\color{mylilas},
    commentstyle=\color{mygrey},%
    showstringspaces=false,%without this there will be a symbol in the places where there is a space
    numbers=left,%
    numberstyle={\ttfamily\color{black}},% size of the numbers
    numbersep=9pt, % this defines how far the numbers are from the text
    emph=[1]{for,end,break},emphstyle=[1]\color{red}, %some words to emphasise
    %emph=[2]{word1,word2}, emphstyle=[2]{style},    
    xleftmargin=\xmrgn,
    framexleftmargin=0.5\bfem
}

\addlitjlstrnum{'all'}{\textquotesingle all\textquotesingle}{5}
\addlitjlstrnum{'X'}{\textquotesingle X\textquotesingle}{3}
\addlitjlstrnum{e7}{e7}{2}
\addlitjlbase{2dBox}{2dBox}{5}
\addlitjlmacros{@muladd}{@muladd}{7}
\addlitjlmacros{@floop}{@floop}{6}
\addlitjlmacros{@reduce}{@reduce}{7}

%\setmonofont{FreeMono}
\graphicspath{ {./figures/} }

\begin{document}

\pagenumbering{gobble}

\begin{algorithm}
    \begin{algorithmic}[1]
        \Function{relative attractor\ }{$f$, $\mathcal{B}$, $n$}
            
            \For{$i = \left\{ 1,\ \dotsc,\ n \right\}$}
                \State $\mathcal{B} \gets$ \Call{subdivide}{$\,\mathcal{B}\,$}
                \State $\mathcal{B} \gets \mathcal{B}\, \cap\, f (\,\mathcal{B}\,)$
            \EndFor

            \State \Return $\mathcal{B}$ 
        \EndFunction
    \end{algorithmic}
\end{algorithm}

\clearpage

\pagebreak

\begin{jllisting}[float, language=matlab, style=jlcodestyle, label=lst:gum:matlab, captionpos=b]
function relative_attractor(t, f, X, n)
dim = t.dim; hit = 1; sd = 8; tic;
for s = 1:n,
    t.set_flags('all', sd);
    t.subdivide(sd);
    b = t.boxes(-1); N = size(b,2);
    S = whos('X'); l = floor(5e7/S.bytes);
    for k = 0:floor(N/l), 
        K = k*l+1:min((k+1)*l,N);
        c = b(1:dim,K);  
        r = b(dim+1:2*dim,1);  
        n = size(c,2); E = ones(n,1);       
        P = kron(E,X)*diag(r) + ...  
            kron(c',ones(size(X,1),1));
        t.set_flags(f(P)', hit); 
    end
    t.remove(hit); 
    fprintf(...
        'depth %d, %d boxes, %.1f sec\n',...
        t.depth,t.count(-1),toc...
    );
end
\end{jllisting}

\clearpage

\pagebreak

\begin{jllisting}[float, language=julia, style=jlcodestyle, label=lst:gum:julia, captionpos=b]
function relative_attractor(f::BoxMap, B::BoxSet, n)
    for s in 1:n
        B = subdivide(B)
        B = B ∩ f(B)
    end
    return B
end
\end{jllisting}

\end{document}
